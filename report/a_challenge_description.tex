\section{Challenge Description}
\label{sec_a}

\subsection{What are the aspects of the challenge you already know a lot about?
What are your assumptions?}

\paragraph{Privacy}

Our aspect is not specialized  on application layer's  user's privacy,  
but we are focusing on privacy transparency on IP and Transport layers; 
specifically on  network traffic generated by the users, and  handled by the ISPs.
An Internet Service Provider logs and stores user's traffic that contains 
personal information. 
This kind of information might be used for further analysis by the ISPs or by 
third party entities such as Universities, Researchers, companies and Law enforcers.	
So, whoever is involved in this procedure should respect the user's privacy by 
applying data protection techniques, such as anonymization in those datasets.

\paragraph{Routing and storage}

Every Autonomous System has its own routing policies to forward traffic and these 
policies are mostly kept in private. ASes can have public and private peering 
agreements. These peering connections can be direct or can be through Internet 
Exchange Points. In the case of Internet Exchange Points, there are two main 
sources to extract information about peering on IXPs. The first source is the 
PeeringDB and the second one is the Packet Clearing House (PCH). 
These sources provide information about the members of each IXP and also provide
the peering policy of each AS. 

Moreover, a lot of research has been conducted to discover the AS topology map. 
CAIDA provides a variety of datasets about ASes and Internet topology. Also there 
are many well-known methodologies to discover topologies like ``DoubleTree'' algorithm 
\cite{caida}. 
These techniques can be used  to validate our results and also provide insighs 
on the routing policies.
Furthermore, big websites, like Facebook, use multiple servers in different locations 
to handle the user traffic. Content is served to users based on their location and 
also based on the load of each server in that time. So we know that our data is 
replicated in different sites (servers) and possible different countries. 

According to the paper ``Internet Surveillance and Boomerang Routing : A Call 
for Canadian Network Sovereignty'' of the 1280 Canada-US-Canada  boomerang  
routes housed in the IXmaps database, data packets are passing through potential 
surveillance cities such as New York, Chicago etc. By doing so, they are 
subjected  to U.S. surveillance efforts. Current findings reveal that cities 
closest to the Canadian border are more likely to route Canadian data. 
Approximately 50\% of IXmaps boomerang routes pass  through  New  York  and 
Chicago, the top two cities suspected of having NSA splitters, and about 25\% 
travel through Seattle. 

\paragraph{Surveillance}

It has been proven that NSA is doing surveillance in Internet traffic. There are 
some applications and services developed to conduct this analysis on a daily 
basis on a huge amount of data. It is known that a large amount of Internet 
servers are located in the US. This is a major reason why internet traffic 
boomerangs through the US. Moreover, it is possible for NSA or other 
intelligence agencies to access that data.

\paragraph{Law enforcement and regulations}

The use of legal powers of each government, in the context of today¢s far more c
complex electronic communications, has proven to be highly controversial. All 
governments have incorporated national security exceptions into national 
legislation to give legal powers to agencies and authorities. Some governments 
have constrained those powers to limit the human rights impact; others have 
created much wider-ranging powers with substantially greater ones. Meanwhile, 
agencies and authorities have the scope to apply advanced analytics techniques 
to every aspect of an individual¢s communications, movements, interests and a
associations - to the extent that such activity is lawful - yielding a depth of 
real-time insights into private lives unimaginable two decades ago.

\paragraph{Economy and advertising}
\footnote{In this paragraph, we discuss user data from a higher level point of 
view.}

It is a fact that personal data are being used for advertising reasons.
Personal data and information about the user's online behavior worth a lot for
companies. The majority of online providers and sellers follow targeted
marketing and advertising  techniques to boost their sales, drawing on
important personal data. Trackers collect information about each individual
customer. Thus, providers are empowered to extract valuable personalized results
and form a marketing technique, targeted on each specific consumer.
In addition, it is also known that service providers use complex algorithms
in order to apply pricing policies to specific groups of consumers.
For instance, Amazon is a well-known provider that follows this merchandise
approach \cite{chen2016empirical}.

Ideally, personalization offers many advantages to the consumers, since it focus
on the consumers' interests. Still, the consumers do not have any control over
their exposed data. To be more specific, one very interesting and crucial
outcome of this personalization is price discrimination. Price discrimination is
a microeconomic pricing strategy, where identical or largely similar goods or
services are transacted at different prices by the same provider in different
markets \footnote{Wikipedia's definition on price discrimination.}.
The data that trackers gather from the users are cross correlated to maximize
the personalization. In fact, there are many works that address this subject.
For example, Ghostery \cite{ghostery} is a browser extension that allows users
to block trackers. Also, AdBlock \cite{adblock}, another browser extension,
helps users surfing without receiving annoying ads. Tor \cite{syverson2004tor}
is another obfuscation technique that directs Internet traffic through a
worldwide distributed network consisting of thousand relays to conceal users¢
location from network monitoring and traffic analysis.

Another controversial issue is the ownership of the user's personal data.
An example could be Facebook purchasing WhatsApp in 2014. The users of WhatsApp 
that agreed to share their data with the company, have had to face the issue of 
the ownership of their data. Recently, in the "terms of service" section of the 
WhatsApp's website has been added the following statement: ``If you are an 
existing user, you can choose not to have your WhatsApp account information 
shared with Facebook to improve your Facebook ads and products experiences'' 
\footnote{\url{https://www.whatsapp.com/legal/?l=en\#key-updates}}.

\subsection{What are the aspects of the challenge you know nothing or little 
about?}

\paragraph{Surveillance locations}

From leaked documents and information we have found that NSA has at least 6 
surveillance locations. It is certain that they will be more of them but there 
is little information. Governments and intelligence agencies are cooperating to 
ensure that this kind of information will remain secret and under the radar.

\paragraph{What happens with routing and storage of user data packets}

We don't know the exact peering matrices inside IXPs. The PeeringDB and PCH 
databases provide information about the membership of each IXP and the routing 
policy of each AS. They don¢t provide any information on who peers with whom i
inside the IXP. Furthermore, the accuracy of the databases is not known because 
they are based on self report and don't provide a strong ground truth. Moreover, 
we can detect only Internet Exchange Points that use layer-3 routing and their 
IPv4/IPv6 prefixes are known. We cannot detect IXPs that operate on layer-2 or 
operate on layer-3 and use private IP prefixes. 

Finally, we don't know exactly where the user data is stored. The user data may 
be replicated in different servers across different countries and these 
information is kept private.

\paragraph{Communications between ISPs and the countries}
Companies and ISPs don't inform the users about the physical locations of data 
storage facilities or what jurisdictions those facilities falls under. Also is 
common practice among ISPs and companies not to disclose what kind of 
information is stored and for what purpose, making difficult to understand if 
users privacy is at risk.

\paragraph{Ethics}
Ethically, the personal data should not be published. The extend of a user's 
wish to share private information should not interfere with another user's 
choice of data sharing. For example, in some cases law enforcement has the 
ability to collect and use certain information when they are investigating 
crimes or prosecuting alleged wrongdoers. The military  should  be able to 
thwart attacks against us. In order to do that, government organizations might 
need to invade some people's privacy in order to uncover illegal acts. 


\paragraph{Communications between ISPs and the countries}
There is a small amount of information regarding ISPs and the traffic they 
exchange in order to serve every user¢s needs. In addition, what is happening to t
the user's traffic which passes through them is unknown, as ISPs tend to not 
publish their sharing policies of users data traffic. There is no information on 
how long user data remains within major ISPs because they are not legally 
required to disclose that information. In recent years, many leaks have come to 
light revealing collaborations between ISPs and governments to collect internet 
communication data. Leaving the various leaks aside there is no way to know the 
agreements between ISPs and the governments which are involved. 

%\paragraph{User trust (from the aspect of user)}
%It is assumed that users do not care about ISPs handling of their information.
%https://gigaom.com/2011/08/11/if-you-cant-trust-your-isp-who-can-you-trust/
%http://www.pcworld.com/article/241591/faq_will_your_isp_protect_your_privacy_.html

\subsection{What are the difficulties that you think you will face in the 
process of working on finding out the information you don't know?}

\paragraph{Surveillance information}
As previously mentioned, the load of traffic traveling through the web is 
enormous. Thus, someone would easily assume that third parties would be 
benefitted by the surveillance of that data to serve their own needs. However, 
proving something like this is not an easy task, since in general, surveillance 
is controversial due to privacy related reasons. Moreover, Internet Service 
Providers and other third party companies that are supposed to be involved on 
this do not tend to offer this information to the public. So, we are not able to 
verify our assumptions.  

\paragraph{Legal problems about finding certain information}
When our data is traveling via the web we can not be aware of the law or 
regulations that are applied to them. 

\paragraph{Leaks on personal info}
Regarding leaks, we can never be aware of where, when and how our data can be 
monitored, stored or used. ISPs or companies that are responsible for routing 
our traffic tend to hide such incidents. Thus, we can not find proofs regarding 
this kind of leaks. 

\paragraph{Ground truth information and validation mechanisms}
Using the term ``ground truth information'', we refer to information that is 
publicly available and acknowledged. In the case of personal data leaks, 
interceptions and surveillance, ground truth information can never be fully 
obtained, due to ISPs and companies not providing users with adequate 
information. 

