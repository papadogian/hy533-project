\section{Challenge Description}
\label{sec_a}

\subsection{What are the aspects of the challenge you already know a lot about?
What are your assumptions?}

\paragraph{Privacy}
Our topic is not specialized  on application layer's  user's privacy,  but we 
are focusing on privacy transparency on lower layers and specifically on traffic 
handled by the ISPs.
It is assumed that the ISP respects users personal information, and the 
conducted experiments and  its results are not revealing any kind of personal 
data (the datasets are stored  and processed after being anonymised). 
Furthermore, ISPs often collaborate with universities and research institutes 
and make the data publicly available.

\paragraph{Routing and storage}
So is our packets safe to the Internet world? Where do they travel and are they 
stored for later analysis? Moreover, the routing policies that ASes use to 
forward our packets are not known. Already, a lot of research has been conducted
to expose the AS and Internet topology. There are many well-known methodologies 
to discover topologies etc. ``DoubleTree'' algorithm. These techniques can be 
used  to validate our results.

\paragraph{Surveillance}
It has been proven that NSA is doing surveillance in Internet traffic. There are
some applications and services developed to conduct this analysis on a daily 
basis at a huge amount of data. It is known that a large amount of internet 
servers are located in US and that is one major reason why internet traffic 
boomerangs through the US. Moreover, it's possible for NSA or other intelligence
agencies to access the data.

\paragraph{Ethics}
Ethically, the personal data should not be published. The extend of a user's 
wish to share private information should not interfere with another user's 
choice of data sharing.  For example, in some cases law enforcement has the 
ability to collect and use certain information when they are investigating 
crimes or prosecuting alleged wrongdoers. The military  should  be able to 
thwart attacks against us. In order to do that, government organizations might 
need to invade some people's privacy in order to uncover illegal acts. 

\paragraph{Law enforcement and regulations}
The use of legal powers of each government, in the context of today's far more 
complex electronic communications has proven to be highly controversial. 
All governments have incorporated national security exceptions into national 
legislation to give legal powers to agencies and authorities. Some governments 
have constrained those powers to limit the human rights impact; others have 
created much wider-ranging powers with substantially greater human rights 
impacts. Meanwhile, agencies and authorities have the scope to apply advanced 
analytics techniques to every aspect of an individual's communications, 
movements, interests and associations – to the extent that such activity is 
lawful – yielding a depth of real-time insights into private lives unimaginable 
two decades ago.

\paragraph{Economy and advertising}
\footnote{In this paragraph, we discuss user data from a higher level point of 
view.}

It is a fact that personal data are being used for advertising reasons. 
Personal data and information about the user's online behavior worth a lot for 
companies. The majority of online providers and sellers follow targeted 
marketing and advertisement techniques to boost their sales, drawing on 
important personal data. Trackers collect information about each individual 
customer. Thus, providers are empowered to extract valuable personalized results 
and form a marketing technique, targeted on each specific consumer.
In addition, it is also known that service providers use complex algorithms
in order to apply pricing policies to specific groups of consumers.
For instance, Amazon is a well-known provider that follows this merchandise 
approach \cite{chen2016empirical}. 

Ideally, personalization offers many advantages to the consumers, since it focus
on the consumers' interests. Still, the consumers do not have any control over 
their exposed data. To be more specific, one very interesting and crucial 
outcome of this personalization is price discrimination. Price discrimination is 
a microeconomic pricing strategy, where identical or largely similar goods or 
services are transacted at different prices by the same provider in different 
markets \footnote{Wikipedia's definition on price discrimination.}.
The data that trackers gather from the users are cross correlated to maximize 
the personalization. In fact, there are many works that address this subject. 
For example, Ghostery \cite{ghostery} is a browser extension that allows users
to block trackers. Also, AdBlock \cite{adblock}, another browser extension,  
helps users surfing without receiving annoying ads. Tor \cite{syverson2004tor} 
is another obfuscation technique that directs Internet traffic through a 
worldwide distributed network consisting of thousand relays to conceal users’ 
location from network monitoring and traffic analysis. 

Another controversial issue is the ownership of the user's personal data.
An example could be Facebook purchasing WhatsApp in 2014. 

\subsection{What are the aspects of the challenge you know nothing or little 
about?}

\paragraph{Surveillance locations}
From leaked documents and information we have found that NSA has at least six 
surveillance locations. It is certain that they will be more of them but there 
is little information. Governments and intelligence agencies are cooperating to 
ensure that this kind of information will remain secret and under the radar.

\paragraph{What happens with routing and storage of user data packets}
We don't know the peering matrices inside IXPs. Also, we can't detect all the 
IXPs due to limited knowledge of their IPv4/v6 prefixes.
Companies and ISPs don't inform the users about the physical locations of data 
storage facilities or what jurisdictions those facilities falls under.

\paragraph{Communications between ISPs and the countries}
ISPs tend to not publish their sharing policies of users data traffic.

\paragraph{User trust (from the aspect of user)}
It is assumed that users do not care about ISPs handling of their information.

\subsection{What are the difficulties that you think you will face in the 
process of working on finding out the information you don't know?}

\begin{enumerate}
\item Surveillance info
\item Legal problems about finding certain informations
\item Leaks on personal info
\item Ground truth information and validation mechanisms
\end{enumerate}
