\section{Research Planning and Deployment}
\label{sec_b}

\subsection{Identify 3 analogous examples you can take inspiration from and 
which elements.}

\paragraph{Tails OS.}
\footnote{\url{https://tails.boum.org/}}

Tails is a live  operating system that aims to preserve privacy and anonymity. 
It helps you to use the Internet anonymously and circumvent censorship almost 
anywhere you go and on any computer, leaving no trace unless you ask it to 
explicitly. Tails relies on the Tor anonymity network to protect your privacy 
online: all software is configured to connect to the Internet through Tor.
If an application tries to connect to the Internet directly, the connection is 
automatically blocked for security.

\paragraph{OpenVPN.}
\footnote{\url{https://openvpn.net/}}

OpenVPN is an open-source software application that implements virtual private 
network (VPN) techniques for creating secure point-to-point or site-to-site 
connections in routed or bridged configurations and remote access facilities.
VPN enables users to send and receive data across shared or public networks as 
if their computing devices were directly connected to the private network. 
Applications running across the VPN may therefore benefit from the functionality, 
security, and management of the private network.
VPNs are  also used to securely connect geographically separated offices of an 
organization, creating one cohesive network. Individual Internet users may 
secure their  wireless transactions with a VPN, to circumvent geo-restrictions 
and censorship, or to connect to proxy servers for the purpose of protecting 
personal identity and location. However, some Internet sites block access to 
known VPN technology to prevent the circumvention of their geo-restrictions.

\paragraph{Princeton Web Transparency \& Accountability Project}
\footnote{\url{https://webtap.princeton.edu}}

Webtap research team, monitors websites and services to find out what user data 
companies collect, how they collect it, and what they do with it. With their 
measurement platform, they study privacy, security, and ethics of consumer data 
usage.

Webtap team has developed  OpenWPM, a generic platform for online tracking 
measurement. It provides the stability and instrumentation necessary to run many 
online privacy studies. It has already been used in several published studies 
from multiple institutions to detect and reverse engineer online tracking.
OpenWPM  is possible to detect and measure many of the known privacy violations 
reported by researchers so far: the use of stateful tracking mechanisms, browser 
fingerprinting, cookie synchronization, and more.

\subsection{In which context would you immerse in order to understand the design 
challenge better?}

The Open Observatory of Network Interference 
\footnote{\url{https://ooni.torproject.org}} is a free software project under 
the Tor Project which aims to detect internet censorship, traffic manipulation 
and signs of surveillance around the world through the collection and processing 
of network measurements. Immersing in those measurements helps us identify 
various number of transparency and privacy issues in the internet.  
A basic privacy issue is the blocking of websites  access for a group of users. 
The main reasons for blocking websites are user censorship and illegal 
activities. Reports generated by Ooni  on blocked websites clarify these facts. 
Also the detection of the systems responsible for censorship and surveillance 
has a big impact on users privacy.

The Tor network \footnote{\url{https://www.torproject.org}} is a group of 
volunteer-operated servers that allows people to 
improve their privacy and security on the Internet. Tor's users employ this 
network by connecting through a series of virtual tunnels rather than making a 
direct connection, thus allowing both organizations and individuals to share 
information over public networks without compromising their privacy. Along the 
same line, Tor is an effective censorship circumvention tool, allowing its users 
to reach otherwise blocked destinations or content.
Using Tor protects you against a common form of Internet surveillance known as 
``traffic analysis''. Traffic analysis can be used to infer who is talking to 
whom over a public network.
A basic problem for the privacy minded is that the recipient of your 
communications can see that you sent it by looking at headers. So can authorized 
intermediaries like Internet service providers, and sometimes unauthorized 
intermediaries as well.
Knowing this problem providing a usable anonymizing network on the Internet 
today is an ongoing challenge. 


