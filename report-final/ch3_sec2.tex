In the first section, three different ideas were provided. The first is the one 
chosen as the best idea and consequently the other two were discarded. The 
chosen one is the only that meets all three of the mentioned criteria; 
(i) innovation, (ii) relevance and (iii) feasibility. Also, it was the one that 
remained after contradicting all three solutions. 
The second solution, though both relevant and feasible, it was not innovative, 
as a similar video has already been published a while ago. As for the third 
solution, it was accounted as both innovative and relevant, but it lacked 
feasibility, as it had the potential to be developed to a certain extend, but 
not completely.

Consequently, after further analysis, the first idea was anointed the best one. 
The extend this idea meets the requested criteria will be further described 
below in sections:
\paragraph{Innovation:} Doing some research, no results popped up that aim to 
provide a service such as the one we proposed. There is no related work in the 
direction of altering the terms and agreements of online services in any way.
\paragraph{Relevance:} By helping users understand the purposefully 
difficult-to-read documents, a higher level of transparency is achieved, as they 
might be able to determine where their information may be forwarded in external 
services or how publicly exposed their data are. 
\paragraph{Feasibility:} In the wide notion of an open source community, this 
service is feasible, as every individual is able to contribute to the extension 
of such service in two ways. The first is to expand the database of keywords 
that are most commonly used in terms and agreements documents as the laws 
evolve, in order to facilitate as much different keywords as possible. The 
second way is to review as much different terms and agreements documents as 
possible, as well as re-reading the already reviewed ones as laws change, in 
order to detect any variations.

