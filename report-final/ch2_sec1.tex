\subsubsection{Questionnaire findings}

The first finding was that, the majority  of the respondents feel at most 
moderately safe on the Internet as 41.5\% of them chose the typical answer. 
Another result from the questionnaire was that internet users believe that the 
protection of their information is not very strict with a percentage of 43.1\%. 
However, even if users believe that there isn't any stringency to the user 
information on the Internet, one in two users do care about their privacy. 
One third of the users have basic knowledge on data transparency and a 
percentage of 47.7\% is aware that their personal information might be leaked on 
the internet even though they are not hacked. 

Concerning whether users care where the companies store their data, the answers 
were divergent. Furthermore, users  tend to care how their traffic is routed 
through the Internet being indicated by the fact that half of them gave a 
corresponding answer. In addition, users care a lot about the identity of the 
third party companies and ISPs through which their traffic is routed. 
A very high percentage of the respondents are being aware that their data can 
traverse through multiple countries. 

Although, they know that their data are traversing multiple countries they are 
not fully aware that legislations of these countries may apply on them, as the 
findings are divergent. A small percentage of the users, specifically 4,6\%, are 
highly informed about the impact of the internet laws on their data. 
On the other hand, approximately 50\% of users care a lot about  the information 
shared by their ISP's to law enforcement agencies and government. Moreover, 
another finding states that very few of the users pay attention to the terms and 
agreements of the Internet Service Providers and online services.

More than 50\% of the responders are fully aware of the existence of targeted 
ads on the Internet, and the role of profiling and tracking entities. On the 
other hand, the answers on how aware of the way the targeted ads work are 
varying and towards high. Their treatment as products tend to concern users a 
lot at a percentage of 40\%. 

A diversity of the users trusting the Internet Service Providers tends to be 
pretty low. This leads users to not trusting the privacy of their information on 
social media at a percentage of 50\%. Regarding the anonymization tools users 
tend to be aware that they exist. In addition, the 20\% are fully aware of their 
existence. Finally, concerning whether the users are willing to pay for 
anonymization tools and privacy services, the answers are contradictory, but 
display a positive attitude towards it.

\subsubsection{Interview findings}

The first finding of the interview is that despite the existing laws, there is 
not a single authority in charge of enforcing them. It is established by the law 
that a warrant is mandatory, for someone to acquire access to  user's data. 
However, there is not a responsible authority to impose penalties in case of any 
illegality.

In addition, we found out that ISPs indeed scan and utilize the incoming and 
outcoming traffic with various mechanisms. Those mechanisms include but are not 
limited to splitters, tapping, NetFlow and SFlow.

Regarding data storage mechanisms, ISPs store data in order to ensure the user's 
security, by detecting and preventing malicious activities. The data are not 
used for advertising reasons.

Moreover, regarding the surveillance there is little information about the exact 
location of the tapping. Surveillance locations, there are not enough 
information revealing their locations, although some insight can be inferred 
from leaked documents. There are submarines and robots, built by the army, that 
act as surveillance mechanisms and are more difficult to be discovered and 
prevented.  There are many researchers working on how to discover the peering 
matrices inside IXPs. Furthermore, there are databases providing this kind of 
information but they might be inaccurate due to the continually change of IXP 
member list.

