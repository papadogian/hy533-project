\section{Creating an action plan.}

\subsection{What core skills would you need in your team in order to execute 
your concept/ plan?}

Our team consist of two core parts.
The first one is the technical team, which contains the web developers 
responsible for the implementation of the web tool. The database will be 
operated by the back-end developers and the whole system will be maintained by 
the administrators and moderators. The latter are also responsible for the right 
functionality of the social part of the community (social media, forums, etc.).
Finally the UI designers will design all the needed web interfaces.

The second team is the economy and advertising group, which involves the economy 
experts, accountant. The economy experts are responsible for the evaluation of 
product costs and setting the goals in the direction of raising funds. On the 
other hand, the accountant is responsible for the distribution of money to each 
member of the team evaluating the economic worth of their work. Also, the 
accountant keeps track of the revenues and expenses in order to identify the 
needs of the project and the people that work for each development.

Finally, the manager is the key person behind the functionality and coordination 
of all the groups. The manager is the equal of a group leader, with exceptional 
people skills in order to motivate the team members and help the product rise. 
A good manager distributes the work equally, taking into consideration the 
capabilities of each individual in order to finally achieve the best possible 
result.

\subsection{What partnerships would you need with other organizations in order 
to support your implementation?}

On every breakthrough idea partnerships are required in order for the idea to be 
supported. Those partnerships are divided into three categories.
First, support is needed from a marketing company. A good marketing agency may 
be able to help e-commerce businesses grow more quickly and alleviate stress for 
the entrepreneurs who own those businesses. It is common for online services and 
extensions to have Facebook pages, Twitter accounts, a YouTube or Google+ 
account, and maybe even an Instagram profile. This leads to an advertisement of 
the given service and resources to reach its full potential. By employing a 
marketing company that specializes in social media can save a significant amount 
of time and, possibly, generate better results.

Another partnership required is with a law firm. This is a partner, who can help 
deal with the difficulties in understanding the glossary used to form the terms 
and conditions document. An issue that is going to be resolved with this 
partnership is comprehension from the perspective of the user regarding the 
effects of accepting  terms and conditions of an online service.

Last but not least, our service needs a group of linguistic consultants. This 
group will help the service develop and grow to all possible users of each 
nationality. In addition, the linguists will help to redefine all the 
information that is going to be provided by the law firm in order to break it 
down to the level of users to understand what they are reading.

\subsection{What is your revenue and cost model? I.e. where will your revenues 
come from and what are your main cost categories? Where will you find funding 
from?}

\paragraph{Funding}
First of all the project will be self funding from the team in order to create a 
basic demo implementation, that will be used for fund raising from crowd-funding 
sites. Crowd-funding platforms like Kickstarter and gofundme that 
gave life to many other projects like Pebble Time that raised over \$20.000.000 
from around 80.000 backers. That funding is needed for the operational costs 
like databases, site hosting, promotion and marketing. From that point forward 
the plug-in will be sustained from donations from users either if they are 
engaged with the community or not.

\paragraph{Costs}
The primary costs for the plug-in are the databases needed and the hosting for 
the community web servers. The code for the plug-in will be open source and 
public at github so anyone can contribute, removing the costs for the developers 
of any kind. As the project will be open source the different teams will not be 
paid directly by the project as they will contribute to the community as 
volunteers. Depending on the amount of funds raised, some specialists in 
advertising, promoting and legal acts will be hired.

\paragraph{Revenue}
As mentioned above the goal of the plug-in is to raise the awareness of the 
users and not to create a business. The revenue will be based on user donations, 
in order to cover the maintenance costs. For that reason the main goal of the 
community is to reach every user of the Internet by making them understand the 
impact that terms and agreements have on their privacy. Also, by allowing 
everyone to contribute to the expansion of the project can transform it to a 
more complete platform. This would offer a greater level of web transparency, 
engaging more users to the community and also sensitizing more to join our cause 
and help by donating.  

