\section{Present your prototype feedback and its evaluation. Present the GOOD, 
the BAD and the UNEXPECTED feedback you received.}
\label{s_7}

\subsection{User evaluations}

\paragraph{User evaluation 1: }
The first user was quite technically savvy and when we presented the prototypes 
to him he had no trouble traversing through the different elements. He 
understood the concept well and he also quoted ``Ah! This is very helpful!''.
\paragraph{User evaluation 2: }
The second user had quite similar educational background. Upon testing the 
product, he had no trouble understanding how to use it, but mentioned it was a 
bit overwhelming when all the comments popped up at once.
\paragraph{User evaluation 3: }
The third user was from a completely different background than the previous two. 
It was noticed that it was more difficult to him to understand the concepts, but 
managed to finish the test in an affordable cost of time. Upon the completion of 
the test, he mentioned that even though the document was easier to understand, 
it still did not give him enough motive to start paying attention to the terms 
and agreements, because he consider it a boring and time consuming task anyway.
\paragraph{User evaluation 4: }
This user found it quite confusing that all the comments popped up at the same 
time and proposed alterations, such as providing the comments one at a time or 
one after another, but make them not to occupy the full page.
\paragraph{User evaluation 5: }
The last user traversed quite easily through the elements, even though he was 
not especially accustomed to such services. He commented that the idea was 
something that he himself would use in his everyday online activities and found 
it very helpful.

\subsection{Feedback outcome}

\paragraph{The good: }
The majority of the users found our proposed add-on prototype very helpful and 
they would be willing to use it in the future. They pointed out that our idea 
was very interesting and significant, since ignoring the terms and agreements, 
nowadays, can prove to be very dangerous regarding the user's privacy. In 
addition, the existence of a wide open community behind this notion provides a 
sense of safety and assurance to the users (continuous updates, discussions, 
support, etc.). 
\paragraph{The bad: }
As expected, the feedback contained also, some negative comments. Fortunately, 
the cons of our prototype were based on the appearance and the functionality of 
our add-on. The majority of the users felt that the comments on the highlighted 
sections might be significantly overwhelming and distracting, when in a large 
amount. 
\paragraph{The unexpected: }
Surprisingly enough, one of the five users that we questioned stated that 
regardless the existence of our add-on and the functionality that it provides, 
he was not willing to use it. 
He specifically mentioned that reading and paying attention to the terms and 
agreements of each service or application is a loss of time. This is something 
very alarming, since it indicates that regardless the privacy preserving tools 
that exist in the wild, there are users that are not even interested in 
attempting using them.

