\section{Plan your prototype test: Who you will prototype with, how and when?}
\label{s_6}

For the prototype testing, we chose to ask five different people. This group of 
potential users is between 20 and 25 year-old university students, mixed female 
and male and derive from a diverse educational background. 
For our testing, we invited the selected group to the university study room, 
where the prototype test took place. We firstly introduced the users with our 
concept and presented the idea. As a second step we asked them to traverse 
through the prototypes and speak their thoughts aloud. Concurrently, we measured 
the time each one needed to understand how to use the system and understand the 
presented results of each action. After the test, we asked them individually to 
elaborate on their thoughts, what they particularly found interesting or what 
confused them and also encouraged them to propose alterations as they wished. 
Furthermore, the university study room seemed to be the perfect location for our 
test, since it was easily accessible for the students. In addition, the place 
was considered ideal and free of distractions. 
The meeting with the group was arranged at the end of the week, since the group 
of those people had a more flexible schedule.

